\documentclass{article}
\usepackage[margin=1in]{geometry}
\usepackage{amsmath, amsfonts, amssymb,venndiagram}
\usepackage{hyperref}
\hypersetup{
    colorlinks=false,
    linktoc=all,
}
    
\title{University of the Pacific ECPE 127 Notes}
\author{}

\begin{document}
\maketitle
\tableofcontents
\newpage
\section{Unit 1 - Probability Theory}
\subsection{Introduction to Set Theory}
A \textbf{set} is a collection of things. For example: 
\begin{itemize}
    \item Collection of all natural numbers $\mathbb{N} = \{1,2,3,4,5,...\}$
    \item Even natural numbers less than or equal to 6: $E = \{2,4,6\}$.
\end{itemize}
Elements of a set are denoted using lowercase letters. For example if $x=4$ belngs to $E$ we would denote that with $x \in E$. To denote an element is \textbf{not} in a set you dash the epsilon: $5 \notin E$.
\subsubsection{Set Operations}
\textbf{Set Union:} \newline
The union of two sets $A$ and $B$ is the set of all elements which is either in $A$ or $B$. Behaves similar to logical OR from digital design. Formal Definiton: $A \cup B = \{ x  | (x \in A) \lor (x \in B)\}$.
\newline
\begin{venndiagram2sets}[tikzoptions={scale=0.95}]
\fillA
\fillB
\end{venndiagram2sets}
\newline
\textbf{Set Intersection:} \newline
$A \cap B$ is the intersection of two sets, and contains every element that is both in $A$ and $B$. Behaves similarly to logical AND from digital design. Formal Defintion: $A \cap B = \{ x | (x \in A) \land (x \in B)\}$
\newline
\begin{venndiagram2sets}[tikzoptions={scale=0.95}]
\fillACapB
\end{venndiagram2sets}\newline
\textbf{Set Compliment:}
$A^c$ is the compliment of $A$ and contains every element not in $A$. Behaves similar to logical NOT from digital design. Formal Definition: $A^c = \{x | x \notin A\}$.
\newline
\begin{venndiagram2sets}[tikzoptions={scale=0.95}]
    \fillNotA
\end{venndiagram2sets}
\newline
\textbf{Set Difference:}
$A - B = A \cap B^c$. Contains every element of $A$ that is not in $B$. \newline
\begin{venndiagram2sets}[tikzoptions={scale=0.95}]
\fillOnlyA
\end{venndiagram2sets}

\subsubsection{Other Definitions}
A collection of sets $A_1, ...., A_n$ is \textbf{mutually exlusive} if and only if:
\[
    A_i \cap A_j = \emptyset \hspace{5mm} i \neq j 
\]
A collection of sets $A_1, \dots, A_n$ is \textbf{collectively exhaustive} if and only if 
\[
A_1 \cup A_2 \cup \dots \cup A_n = S
\]
Two sets are equal to each other if and only if 
\[
    (A \subseteq B) \land (B \subseteq A)
\]
\textbf{De Morgan's Law:} De Morgan's law relates all three basic set operations
\[
    (A \cup B)^c = A^c \cap B^c
\]
\textbf{Proof:} Let $x \in (A \cup B)^c$. Then as $x \notin A \cup B$ therefore $x \notin A$ and $x \notin B$ Therefore $x \in A^c \cap B^c$ Therefore $(A \cup B)^c \subseteq A^c \cap B^c$. Now assume $x \in A^c \cap B^c$. Then $x \notin A$ and $x \notin B$, and therefore $x \notin (A \cup B)$. Thus $x \in (A \cup B)^c$. Therefore $(A \cup B)^c = A^c \cap B^c$. $\square$
\[
 (A \cap B)^C = A^c \cup B^c
\]
\textbf{Proof:} Let $x \in (A \cap B)^c$. Then $x$ is either in $A$ not in $B$, in $B$ not in $A$, or not in either $A$ or $B$. Therefore $x \in A^c \cup B^c$ and thus $(A \cap B)^c \subseteq A^c \cup B^c$. Now let $x \in A^c \cup B^c$. Then by definition $x$ is either in $A^c$ or $B^c$. Thus $x$ is not in both $A$ and $B$. Therefore $x \in (A \cap B)^c$ and thus $A^c \cup B^c \subseteq (A \cap B)^c$. As $A^c \cup B^c \subseteq (A \cap B)^c$ and $(A \cap B)^c \subseteq A^c \cup B^c$, $(A\cap B)^c = A^c \cup B^c$. $\square$ 

\subsection{Applying Set Theory to Probability}
An \textbf{experiment} consists of a procedure and observations.

\begin{center}
\begin{tabular}{c|c|c}
    Experiment & Procedure & Observation \\
    \hline
    Coin Flip & Flip the coin & heads or tails \\
    Dice Rolls & Roll the die & the number face up on the die \\
    Networking & Send packets & Record the packets that successfully get transmitted
\end{tabular}
\end{center}
The \textbf{sample space} of an experiment is the finest-grain, mutually exclusive, collectively exhaustive set of all possible outcomes.  

\begin{center}
    \begin{tabular}{c|c}
        Roll a die & $S = D = \{1,2,3,4,5,6\}$ \\
        Flip a coin & $S = C = \{H,T\}$ \\
        Flip a coin twice & $S = F = \{HH, HT, TH, TT\}$
    \end{tabular}
\end{center}
An \textbf{event} is a set of desired outcomes of an experiment. Example: Roll a die, you win if you roll an even number. $E = \{2,4,6\}$.

\subsection{Axioms}
Probability P maps the events from a sample space to real numbers such that
\begin{enumerate}
    \item $P(A) \geq 0$ where $A$ is an event in the sample space $S$
    \item $P(S) = 1$ where $S$ is the universal set 
    \item For a countable collections of mutually exclusive sets $A_1, A_2, A_3, ... A_n \in S$, $P(A_1 \cup A_2 \cup \dots \cup A_n) = P(A_1) + P(A_2) + \dots + P(A_n)$

\end{enumerate}
\subsection{Theorems}
\textbf{Theorem 1.4} \newline
$P(A^c) = 1 - P(A)$. 

For any two sets $A$ and $B$ not necessarily mutually exclusive:
\[
P(A \cup B) = P(A) + B(B) - P(A \cap B)
\]
Visual Explaination:

\begin{venndiagram2sets}
\fillA
\end{venndiagram2sets}

\begin{venndiagram2sets}
\fillB
\end{venndiagram2sets}


Here we see the intersection of $A$ and $B$ could be counted twice if we add $P(A)$ and $P(B)$ so we have to subtract the intersection so it is only counted once. \newline \newline
\textbf{Theorem 1.5} \newline
The probability of event $B =  \{s_1, s_2, \dots s_m\}$ is the sum of probabilities contained in the event:
\[
    P(B) = \sum_{i=1}^{m} P(\{s_i\})
\]
Follows from axiom 3 as each $s_i$ is mutually exclusive.  \newline \newline
\textbf{Theorem 1.6} \newline
For an experiment with sample space $S = \{s_1, \dots, s_n \}$ in which each outcome $s_i$ is equally likely, 
\[
    P(s_i) = 1/n \hspace{5mm} 1 \leq i \leq n
\]
Where $n$ is the number of outcomes in the sample space (same as n is equal to the cardinality of $S$)  \newline
\newline
\textbf{Theorem 1.7} \newline
If outcomes in an experiment are equally likely, then probability of event A is given by:
\[
P(A) = \frac{|A|}{|S|}
\]
Where $||$ denotes the cardinality of the set.
\subsection{Conditional Probability}
\end{document}
